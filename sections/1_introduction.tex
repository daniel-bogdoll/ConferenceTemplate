\section{Introduction}
\label{sec:introduction}

Context of the work. Use \url{https://www.zotero.org/} for your literature management. Every group managed by an FZI employee has unlimited storage for your files. Also, if you want to work offline or in your own IDE, you can simply git clone this project via the menu on the left. Test of all the citations
\cite{Bogdoll_Compressing_2021_NeurIPS}
\cite{Bogdoll_Description_2021_ICCV}
\cite{Bogdoll_KIGLIS_2021_ISC2}
\cite{Bogdoll_Taxonomy_2021_arXiv}
\cite{Toettel_Reliving_2021_arXiv}
\cite{Reichert_Towards_2021_ISC2}
\cite{Bogdoll_Augmenting_2017_US}
\cite{Koduri_Aureate_2018_WCX, Asam_Openscenario_2020_Web}.

If you refer to figures and sections, please use the commands \textit{figref} and \textit{secref} as shown here: \figref{fig:compare_sensors} and \secref{sec:acknowledgment}. If you want to link to your open-source repository, please use a an href: All code is available on \href{https://github.com/daniel-bogdoll/deep_generative_models}{\color{wong-lightblue}{GitHub}}. \\

Find more details on style here: \url{https://github.com/daniel-bogdoll/ConferenceTemplate/blob/main/README.md}. For inspirations, take a look at awarded papers such as here: \url{https://cvpr2021.thecvf.com/node/329}. If you want to include tables, the \url{https://www.tablesgenerator.com/} is a helpful tool. When you use acronyms, the \textit{acronym} package automatically makes sure that every acronym is spelled out the first time you use it: First time: \ac{ad} and second time: \ac{ad}.

\subsection{Contribution}
Gap statement/research question and contribution. Identify the core contribution. Before you start writing anything it’s important to identify \textbf{the single core contribution} that your paper makes to the field. More text here. Code and data is
available at \href{https://github.com/xxxxx}.


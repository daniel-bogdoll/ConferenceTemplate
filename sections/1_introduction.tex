\section{Introduction}
\label{sec:introduction}

Context of the work. Use \url{https://www.zotero.org/} for your literature management. Every group managed by an FZI employee has unlimited storage for your files. Also, if you want to work offline or in your own IDE, you can simply 'git clone' this project via the menu on the left. Test of all the citations
\cite{Bogdoll_DLCSS_2022_ICECCME}
\cite{Bogdoll_Experiments_2022_ICECCME}
\cite{Woermann_Knowledge_2022_arXiv}
\cite{Bogdoll_Situation_2022_Knowledge}
\cite{Rudolph_Reinforcement_2022_Knowledge}
\cite{Bogdoll_Multimodal_2022_SMC}
\cite{Bogdoll_Anomaly_2022_CVPR}
\cite{Bogdoll_Quantification_2022_IV}
\cite{Bogdoll_Addatasets_2022_VEHITS}
\cite{Toettel_Reliving_2022_ICITE}
\cite{Bogdoll_Taxonomy_2022_FICC}
\cite{Bogdoll_Compressing_2021_NeurIPS}
\cite{Bogdoll_Description_2021_ICCV}
\cite{Bogdoll_KIGLIS_2021_ISC2}
\cite{Reichert_Towards_2021_ISC2}
\cite{Koduri_Aureate_2018_WCX}
\cite{Bogdoll_Augmenting_2017_US}

\cite{Guneshka_Ontology_2022_BA}
\cite{Sartoris_Anomaly_2022_BA}
\cite{Schilling_Anomaly_2022_MA}

\cite{Asam_Openscenario_2020_Web}
\cite{SAE_J3016_2021_Standard}
\cite{IEEE_2846_2022_Standard}

If you refer to figures and sections, please use the commands \textit{figref} and \textit{secref} as shown here: \figref{fig:compare_sensors} and \secref{sec:acknowledgment}. If you want to link to your open-source repository, please use a href: All code is available on \href{https://github.com/daniel-bogdoll/deep_generative_models}{\color{wong-lightblue}{GitHub}}. \\

Find more details on style here: \url{https://github.com/daniel-bogdoll/ConferenceTemplate/blob/main/README.md}. For inspirations, take a look at awarded papers such as here: \url{https://cvpr2021.thecvf.com/node/329}. If you want to include tables, the \url{https://www.tablesgenerator.com/} is a helpful tool. When you use acronyms, the \textit{acronym} package automatically makes sure that every acronym is spelled out the first time you use it: First time: \ac{ad} and second time: \ac{ad}.

\textbf{Research Gap.}
Gap statement/research question. A research gap is a question or a problem that has not been answered by any of the existing studies or research within your field. Sometimes, a research gap exists when there is a concept or new idea that hasn't been studied at all. The literature review is a core element of your thesis and shows that you are capable of working scientifically. As you explain what other researchers have found on your topic, the reader will realize that you know this topic extremely well. This will build trust that you can provide a piece of work yourself that is scientifically relevant.

\textbf{Contribution.}
 Contribution. Identify the core contribution. Before you start writing anything, it’s important to identify \textbf{the single core contribution} that your paper makes to the field. Anyone who spends even very little time reading the paper needs to understand the contribution.

